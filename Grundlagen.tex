\section{Grundlagen}
\subsection{Einheiten}
\begin{table}[H]
	\renewcommand{\arraystretch}{2.15}
\begin{tabularx}{0.9\columnwidth}{lXl}
	Größe & Symbol & Einheit\\
	\hline
	Permiabilitätskonstante & $\mu_0$ &  $\dfrac{\texttt{Vs}}{\texttt{Am}}$\\
	\hline
	Dilelektrizitätskonstante & $\varepsilon_0$ &  $\dfrac{\texttt{As}}{\texttt{Vm}}$\\
		\hline
	elek. Ladung/Fluss & $ Q, q $ & $ C=As $\\
	\hline
	elek. Feldstärke & $ \vec{E} $ & $\dfrac{\texttt{V}}{\texttt{m}}$\\
	\hline
	elek. Flussdichte & $ \vec{D} $ & $\dfrac{\texttt{As}}{\texttt{m}^2}=\dfrac{\texttt{C}}{\texttt{m}^2}$\\
	\hline
	Kapazität & $C$ &  $F= \dfrac{\texttt{As}}{\texttt{V}}$\\
		\hline
	mag. Fluss& $\phi, \Phi$ &  $Wb = Vs$\\
		\hline
	mag. Feldstärke & $\vec{H}$ &  $\dfrac{A}{m}$\\
		\hline
	mag. Flussdichte & $\vec{B}$ &  $T = \dfrac{\texttt{Vs}}{\texttt{m}^2}$\\
	\hline
	Induktivität & $L$ &  $H = \dfrac{\texttt{Vs}}{\texttt{A}}$\\
		\hline
	Strahlungsdichte & $S_{av}, I$ &  $\dfrac{\texttt{W}}{\texttt{m}^2}$\\
				
%	A, W & Arbeit, Energie & J = VAs = Ws\\
%	$\vec{A}$ & mag. Vektorpotenzial & $\dfrac{Vs}{m} = \dfrac{T}{m}$ ($\vec{B}$ = $\nabla \times \vec{A}$)\\
%	$\vec{B}$ & mag. Flussdichte & $T = \dfrac{Vs}{m^2}$\\
%	C & Kapazität & $F = \dfrac{As}{V}$\\
%	$\vec{D}$ & dielek. Verschiebung/Erregung & $\dfrac{As}{m^2}$\\
%	e, q, Q & (Elementar-)ladung & C = As\\
%	$\vec{E}$ & elek. Feldstärke & $\dfrac{V}{m}$ \\
%	$\vec{H}$ & mag. Feldstärke/Erregung & $\dfrac{A}{m}$\\
%	$\vec{J}$ & Stromdichte & $\dfrac{A}{m^2}$\\
%	$\vec{J}_F$ & Flächenstromdichte & $\dfrac{A}{m}$\\
%	$\vec{M}$ & Drehmoment & J = Nm = VAs\\
%	F & Kraft & $\dfrac{kgm}{s} = N$\\
%	$R_{mag}$ & mag. Widerstand & $\dfrac{S}{s} = \dfrac{A}{Vs}$\\
%	$\vec{S}$ & Poynting-Vektor & $\dfrac{W}{m^2}$\\
%	Z & Wellenwiderstand & $\Omega$\\
%	$\delta_s$ & Eindringtiefe & m \\
%	$\varepsilon$ & Dielektrizitätskonstante & $\dfrac{As}{Vm}$\\
%	$\varphi$ & elek. Skalarpotenzial & V \\
%	$\varphi_m$ & mag. Skalarpotenzial & A \\
%	$\rho$ & Raumladungsdichte & $\dfrac{As}{m^3}$\\
%	$\rho$ & spez. Widerstand & $\dfrac{\Omega}{m} = \dfrac{VA}{m}$\\
%	$\kappa, \sigma$ & elek. Leitfähigkeit & $\dfrac{S}{m} = \dfrac{A}{Vm}$\\
%	$\lambda$ & Wellenlänge & m\\
%	$\Phi_e$ & elek. Fluss & C = As\\
%	$\Phi_m$ & mag. Fluss & $Wb = \dfrac{T}{m^2}$\\
%	\hline
\end{tabularx}
\end{table}
\subsection{Vektorrechnung}
\subsubsection{Betrag, Richtungswinkel, Normierung}
\textbf{Betrag}
\begin{align*}
	\vert \vec{r}  \vert & = r = \sqrt{r^2_x + r^2_y + r^2_z}
\end{align*}
\textbf{Richtungswinkel}
\begin{align*}
	\cos(\alpha) = \dfrac{a_x}{\vert \vec{a} \vert} \qquad \cos(\beta) = \dfrac{a_y}{\vert \vec{a} \vert} \qquad
	\cos(\gamma) = \dfrac{a_z}{\vert \vec{a} \vert}
\end{align*}
\textbf{Normierung, Einheitsvektor}
\begin{align*}
	\vec{e}_a =  \dfrac{\vec{a}}{\vert \vec{a} \vert}, \quad \vert \vec{e}_a \vert = 1
\end{align*}

\subsubsection{Skalarprodukt} (Schnittwinkel zweier Vektoren)
\begin{align*}
	\vec{a} \cdot \vec{b} & = |\vec{a}| \cdot |\vec{b}| \cdot cos(\varphi) \qquad \vec{a} \cdot \vec{b}  = 0\\
	cos(\varphi)          &  = \dfrac{\vec{a} \cdot \vec{b}}{|\vec{a}| \cdot |\vec{b}|} = \dfrac{a_x \cdot b_x + a_y \cdot b_y + a_z \cdot b_z}{|\vec{a}| \cdot |\vec{b}|}
\end{align*}

\subsubsection{Kreuzprodukt}

\begingroup
\renewcommand*{\arraystretch}{.95}
\begin{align*}
	A_{Para} & = \vert \vec{c} \vert = \vert \vec{a} \times \vec{b} \vert = \vert \vec{a} \vert \cdot \vert \vec{b} \vert \cdot \sin(\varphi)\\
	\vec{a}\times\vec{b} & =
	\begin{pmatrix}
		a_x \\
		a_y \\
		a_z
	\end{pmatrix}
	\times
	\begin{pmatrix}
		b_x \\
		b_y \\
		b_z
	\end{pmatrix} =
	\begin{pmatrix}
		a_yb_z-a_zb_y \\
		a_zb_x-a_xb_z \\
		a_xb_y-a_yb_x
	\end{pmatrix}
\end{align*}
\endgroup
Trick: Regel von Sarrus anwenden!

\subsection{Differentialoperatoren}
\textbf{Nabla-Operator}
\begin{align*}
    \nabla & = \vec{\nabla} = 
    \begin{psmallmatrix}
        \partial  / \partial x \\
        \partial  / \partial y \\
        \partial  / \partial z
    \end{psmallmatrix}
\end{align*}

\textbf{Laplace-Operator}
\begin{align*}
    \varDelta  & = \vec{\nabla} \cdot \vec{\nabla} = \textrm{div (grad)} = 
    \dfrac{\partial ^2}{\partial x^2}+\dfrac{\partial ^2}{\partial y^2}+\dfrac{\partial ^2}{\partial z^2}
\end{align*}

\textbf{Divergenz} $\opdiv$: Vektorfeld $\rightarrow$ Skalar \qquad S.382\\
\small{Quelldichte, gibt für jeden Punkt im Raum an, ob Feldlinien entstehen oder verschwinden.}
\begin{align*}
    \boxed{\opdiv \vec{F} = \nabla \cdot \vec{F}}   =  \dfrac{\partial F_x}{\partial x} 
    + \dfrac{\partial F_y}{\partial y} + \dfrac{\partial F_z}{\partial z}\\ 
                                 \begin{cases}
    > 0 \quad\Rightarrow \textnormal{Quelle}  \\
    < 0 \quad\Rightarrow \textnormal{Senke} \\
    = 0 \quad\Rightarrow \textnormal{quellenfrei} 
\end{cases}                                      
\end{align*}
% Bsp: $\opdiv \vec{B} = 0$, da mag. Felder in sich geschlossen.\\

\textbf{Rotation} $\oprot$: Vektorfeld $\rightarrow$ Vektorfeld \qquad S.382\\ 
\small{Wirbeldichte, gibt für jeden Punkt im Raum Betrag und Richtung der Rotationsgeschwindigkeit an.}
\begin{align*}
\boxed{\oprot \vec{F} = \nabla \times \vec{F} } = 
\begin{pmatrix}
    \dfrac{\partial F_z}{\partial y} - \dfrac{\partial F_y}{\partial z} \\
    \dfrac{\partial F_x}{\partial z} - \dfrac{\partial F_z}{\partial x} \\
    \dfrac{\partial F_y}{\partial x} - \dfrac{\partial F_x}{\partial y}
\end{pmatrix} =
\begin{vmatrix}
    \vec{e}_x & \vec{e}_y & \vec{e}_z \\
    \dfrac{\partial}{\partial x} & \dfrac{\partial }{\partial x} & \dfrac{\partial }{\partial z} \\
    \vec{F}_x & \vec{F}_y & \vec{F}_z
\end{vmatrix}
\end{align*}\\
Vektorfeld skalar annotiert: $\vec{F} = \vec{F}(x;y;z) = F_x\vec{e}_x+F_y\vec{e}_y+F_z\vec{e}_z$\\
% Bsp: energieerhaltende (konservatives) Felder $ \rightarrow \oprot = 0$\\

\textbf{Gradient} $\opgrad$: Skalarfeld $\rightarrow$ Vektor/Gradientenfeld\\ 
\small{zeigt in Richtung steilster Anstieg von $\phi$}
\begin{align*}                                                                                          
    \boxed{\opgrad \phi = \nabla \cdot \phi }=  
    % \hspace{4.5ex}
    \begin{psmallmatrix}
        \partial \phi / \partial x \\
        \partial \phi / \partial y \\
        \partial \phi / \partial z
        % \dfrac{\partial G}{\partial x}\\
        % \dfrac{\partial G}{\partial y}\\
        % \dfrac{\partial G}{\partial z}
    \end{psmallmatrix}
    = \dfrac{\partial \phi}{\partial x} \vec{e}_x + \dfrac{\partial \phi}{\partial y} \vec{e}_y + 
    \dfrac{\partial \phi}{\partial z} \vec{e}_z  
\end{align*}

\subsubsection{Rechenregeln}
$\phi, \psi$: Skalarfelder \qquad $\vec{A}, \vec{B}$: Vektorfelder
\begin{align*}
     & \nabla \cdot (\vec{A} \times \vec{B}) & = & \qquad (\nabla \times \vec{A})\cdot\vec{B} - (\nabla\times\vec{B})\cdot\vec{A} \\
     & \nabla \cdot (\phi \cdot \psi)        & = & \qquad \phi (\nabla \psi) + \psi( \nabla \phi)                                  \\
     & \nabla \cdot (\phi \cdot \vec{A})           & = & \qquad \phi (\nabla \vec{A}) + \vec{A}(\nabla \phi)                             \\
     & \nabla \times (\phi \cdot \vec{A})          & = & \qquad \nabla \phi \times \vec{A} + \phi (\nabla \times \vec{A})                        
    %  & \oprot \opgrad f                      & = & \qquad 0 \Rightarrow\textnormal{Gradientenfeld Quellenfrei}                    \\
    %  & \opdiv \oprot \vec{f}                 & = & \qquad 0 \Rightarrow\textnormal{Wirbelfeld Quellenfrei}
\end{align*}

\subsubsection{Spezielle Vektorfelder}
quellenfreies Vektorfeld $\vec{F}$ $\rightarrow$ Vektorpotential $\vec{E}$
\begin{align*}
\opdiv \vec{F} = \boxed{\opdiv (\oprot \vec{E}) = 0} \quad \Leftrightarrow \quad  \vec{F} = \oprot \vec{E}
\end{align*}
wirbelfreies Vektorfeld $\vec{F}$ $\rightarrow$ Skalarpotential $\phi$
\begin{align*}
    \oprot \vec{F} = \boxed{\oprot (\opgrad \phi) = 0} \quad \Leftrightarrow \quad  \vec{F} = \opgrad \phi
\end{align*}
quellen- und wirbelfreies Vektorfeld $\vec{F}$:
\begin{align*}
    &\oprot \vec{F}  = 0 \quad \opdiv \vec{F} = 0\\
    &\opdiv (\opgrad \phi) = \varDelta \phi = 0 \quad \Leftrightarrow \quad  \vec{F} = \opgrad \phi\\
    &\oprot (\oprot \vec{F})  = \opgrad (\opdiv \vec{F}) - \varDelta \vec{F} 
\end{align*}

% Feldänderung bei Bewegung
% \begin{align*}
%     \Delta G & = \dfrac{\partial G}{\partial x} \Delta x + \dfrac{\partial G}{\partial y} \Delta y + \dfrac{\partial G}{\partial z} \Delta z \\
%              & = dG = \opgrad G \cdot d \vec{s}
% \end{align*}

% \columnbreak


    \subsection{Logarithmische Maße/Pegel}
	Wichtig: Feldgrößen sind \textbf{Effektivwerte}!
        \begin{itemize} 
            \item \textbf{Dämpfungsmaß} $ a $ in Dezibel [dB] und Neper [Np]
            \begin{flalign*}
                1 \, \si{dB} & =  0,1151 \, \si{Np} & 1 \, \si{Np} & = 8,686 \, \si{dB} & \\  
                a \,[\si{dB}]  & = 20 \cdot \log_{} \dfrac{F_1}{F_2} & a \,[\si{dB}]  & = 10 \cdot \log_{}  \frac{P_1}{P_2}& \\
                \frac{F_1}{F_2} & =  10^{\frac{a[\si{dB}]}{20\si{dB}}} & \frac{P_1}{P_2}   & =   10^{\frac{a[\si{dB}]}{10\si{dB}}} &\\
                a \,[\si{Np}]  & = \ln \dfrac{F_1}{F_2} & a \,[\si{Np}]  & = \dfrac{1}{2} \cdot \ln \dfrac{P_1}{P_2} & \\
                \frac{F_1}{F_2} & =  e^{a[\si{Np}]}                   & \frac{P_1}{P_2}   & = e^{2a[\si{Np}]} &
	            \end{flalign*}
        	\item \textbf{absolute Pegel} $ L $ mit Bezugsgrößen $ P_0, F_0 $
			\begin{flalign*}
				L \,[\si{dB}]  & = 20 \cdot \log_{} \dfrac{F_1}{F_0} & L \,[\si{dB}]  & = 10 \cdot \log_{}  \frac{P_1}{P_0}& \\
				\frac{F_1}{F_0} & =  10^{\frac{L[\si{dB}]}{20\si{dB}}} & \frac{P_1}{P_0}   & =   10^{\frac{L[\si{dB}]}{10\si{dB}}} &
			\end{flalign*}
            \renewcommand\arraystretch{1.4}
			\begin{tabularx}{0.8\columnwidth}{l|X|X}
			\hline
			Einheit & Bezugswert & Formelzeichen\\
			\hline
			dBm, dB(mW) & $ P_0 = 1mW $ & $ L_{\texttt{P/mW}}$ \\
			dBW, dB(W) & $ P_0 = 1W $ & $ L_{\texttt{P/W}}$ \\
%			\hline
%			dBV, dB(V) & $ F_0 = 1V $ & $ L_{\texttt{U/V}}$ \\
%			dB$\mu$V, dB($\mu$V) & $ F_0 = 1\mu V $ & $ L_{\texttt{U/$\mu$V}}$ \\
%			dB$\mu$A, dB($\mu$A) & $ F_0 = 1\mu A $ & $ L_{\texttt{I/$ \mu $A}}$ \\
%			dB($ \mu $V/m) & $ F_0 = 1 \tfrac{\mu V}{m} $ & $ L_{\texttt{E/($ \mu $V/m)}}$\\
%			dB($ \mu $A/m) & $ F_0 = 1 \tfrac{\mu A}{m} $ & $ L_{\texttt{H/($ \mu $A/m)}}$\\
			\hline
			\end{tabularx}
			
%            \item \textbf{Umrechnung} (Annäherungswerte)
%            \renewcommand\arraystretch{1.2}
%            \begin{table}[H]
%	            \begin{tabularx}{1\columnwidth}{l|X|X}
%	            	\hline
%            		Faktor $ \tfrac{F_1}{F_0} \, \text{bzw.} \, \tfrac{P_1}{P_0}$ & Energiegröße $P_n$ & Feldgröße $F_n$\\
%            		\hline
%            		1 & 0 & 0 \\
%            		100 & 20 \si{dB} & 40 \si{dB} \\
%            		1000 & 30 \si{dB} & 60  \si{dB} \\
%            		0,1 & -10 \si{dB} & -20 \si{dB} \\
%            		0,01 & -20 \si{dB} & -40 \si{dB} \\
%            		0,001 & -30 \si{dB} & -60 \si{dB} \\
%            		2 & 3 \si{dB} & 6 \si{dB} \\
%            		4 & 6 \si{dB} & 12 \si{dB} \\
%            		8 & 9 \si{dB} & 18 \si{dB} \\
%            		0,5 & -3,01 \si{dB} & -6,02 \si{dB} \\
%            		1,25 & 0,97 \si{dB} & 1,94 \si{dB} \\
%            		0,8 & -0,97 \si{dB} & -1,94 \si{dB} \\
%            		\hline
%            	\end{tabularx}    
%            \end{table}
        \item \textbf{relativer Pegel / Maß}\\
        Maß = Differenz zweier (Leistungs)pegel bei\\ gleichem Bezugswert $ P_0 $
        \begin{equation*}
        	\Delta L = L_2 - L_1 = 10 \cdot \log \left( \frac{P_2}{P_1}\right)  \si{dB}
        \end{equation*}
        \end{itemize}
%\textit{Pegel}: feste Bezugsgröße $ P_0 $\\
%\textit{Maß}: beliebige Bezugsgröße $ P_2 $ bzw. Differenz zweier Pegel\\


% \subsection{Begriffe}
% \begin{tabularx}{0.45\textwidth}{>{\hsize=.1\hsize}X|>{\hsize=.5\hsize}X|>{\hsize=.4\hsize}X}
%            & Begriff           & Beschreibung \\
%     \hline
%     $\rho$ & Raumladungsdichte &              \\
% \end{tabularx}

\subsubsection{Rechnen mit Pegeln / Logarithmen}
Rechenregeln für Logarithmen (10er-Basis): \quad $ x,y,a > 0 $
\begin{flalign*}
	\log (x\cdot y) &= \log (x) - \log (y) & \log (\tfrac{x}{y}) & = \log (x) - \log (y) &\\
	\log (x^a) &= a\cdot \log(x)  & \log \sqrt[a]{x} & = \frac{1}{a} \cdot \log (x) &\\
   \mathtt{Pegel} &= 10 \cdot \log(\mathtt{Faktor}) & \mathtt{Faktor}& = 10 ^{\tfrac{\mathtt{Pegel}}{10}}
\end{flalign*}

\subsection{Rechnen mit Wurzeln}
$a$: Radikant \qquad $n$: Wurzelexponent\\
Merke: $\sqrt[n]{a \pm b} \neq \sqrt[n]{a} \pm \sqrt[n]{b}$ \qquad $ x = \sqrt[n]{a} = a^{\frac{1}{n}} $
\begin{flalign*}
&\sqrt[n]{a^m}=\left(a^m\right)^{\frac{1}{n}}=\left(a^{\frac{1}{n}}\right)^m=a^{\frac{m}{n}}=(\sqrt[n]{a})^m\\
&\sqrt[m]{\sqrt[n]{a}}=\sqrt[m]{a^{\frac{1}{n}}}=\left(a^{\frac{1}{n}}\right)^{\frac{1}{m}}=a^{\frac{1}{m \cdot n}}=\sqrt[m \cdot n]{a}\\
&\sqrt[n]{a} \cdot \sqrt[n]{b}=\left(a^{\frac{1}{n}}\right) \cdot\left(b^{\frac{1}{n}}\right)=(a b)^{\frac{1}{n}}=\sqrt[n]{a b}\\
&\frac{\sqrt[n]{a}}{\sqrt[n]{b}}=\frac{a^{\frac{1}{n}}}{b^{\frac{1}{n}}}=\left(\frac{a}{b}\right)^{\frac{1}{n}}=\sqrt[n]{\frac{a}{b}} \quad(b>0)
\end{flalign*}

\subsection{Rechnen mit Potenzen}
$a$: Basis \qquad $m,n$: Exponent
\begin{flalign*}
	&a^m \cdot a^n=a^{m+n} & \frac{a^m}{a^n}&=a^{m-n} \quad(a \neq 0)&\\
	&\left(a^m\right)^n=\left(a^n\right)^m=a^{m \cdot n}& a^n \cdot b^n&=(a \cdot b)^n&\\
	& \frac{a^n}{b^n}=\left(\frac{a}{b}\right)^n \quad(b \neq 0)& a^b &= e^{b \cdot \ln a}&\\
	&a^0 =1 & a^{-n} &= \frac{1}{a^n}&
\end{flalign*}


\subsection{Koordinatensysteme}


\subsubsection{Umrechnungstabelle}
\begin{tabularx}{0.45\textwidth}{>{\hsize=.46\hsize}X|>{\hsize=.27\hsize}X|>{\hsize=.27\hsize}X}
    Kart.                                                                                & Zyl.             & Kug.                            \\
    \specialrule{1.5pt}{0pt}{0pt}
    $x$                                                                                  & $r \cos \varphi$ & $r \sin \vartheta \cos \varphi$ \\
    \hline
    $y$                                                                                  & $r \sin \varphi$ & $r \sin \vartheta \sin \varphi$ \\
    \hline
    $z$                                                                                  & $z$              & $r \cos \vartheta$              \\
    \specialrule{1.5pt}{0pt}{0pt}
    $\sqrt{x^{2}+y^{2}}$                                                                 & $r$              &                                 \\
    \hline
    $\arctan \frac{y}{x}$                                                                & $\varphi$        &                                 \\
    \hline
    $z$                                                                                  & $z$              &                                 \\
    \hline
    $d x \cos \varphi+d y \sin \varphi$                                                  & $dr$             &                                 \\
    \hline
    $d y \cos \varphi-d x \sin \varphi$                                                  & $r d\varphi$     &                                 \\
    \hline
    $dz$                                                                                 & $dz$             &                                 \\
    \specialrule{1.5pt}{0pt}{0pt}
    $\sqrt{x^{2}+y^{2}+z^{2}}$                                                           &                  & $r$                             \\
    \hline
    $\arctan \frac{y}{x}$                                                                &                  & $\varphi$                       \\
    \hline
    $\arctan \frac{\sqrt{x^{2}+y^{2}}}{z}$                                               &                  & $\vartheta$                     \\
    \hline
    $d x \sin \vartheta \cos \varphi+d y \sin \vartheta \sin \varphi+d z \cos \vartheta$ &                  & $dr$                            \\
    \hline
    $d y \cos \varphi-d x \sin \varphi$                                                  &                  & $r \sin \vartheta d \varphi$    \\
    \hline
    $d x \cos \vartheta \cos \varphi+d y \cos \vartheta \sin \varphi-d z \sin \vartheta$ &                  & $r d \vartheta$                 \\
\end{tabularx}