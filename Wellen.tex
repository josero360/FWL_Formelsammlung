\section{Wellen}
\subsection{Grundlagen}
Energie- ohne Materietransport

\subsubsection*{Wellengleichung}
\textbf{Tatsächlicher Zeitverlauf:} (Realteil von $\underline{\vec{E}}(z,t)$)
\begin{align*}
    \vec{E}(z,t)
        &= E_0
        \cdot \overbrace{e^{-\alpha z}}^{\mathclap{\text{Dämpfung}}}
        \cdot \underbrace{cos(\omega t \overbrace{-}^{\mathclap{\text{positive z-Richtung}}} \beta z)}_\text{Zeit- \& Raumausb. (Wellenz./S)}
        \cdot \overbrace{\vec{e}_y}^{\mathclap{\text{Feldausb.}}} \\
    \vec{H}(z,t)
        &= - H_0 \cdot cos(\omega t - \beta z) \cdot \vec{e}_x \qquad \text{wenn $Z_F$ reel}
\end{align*}
\vspace{1ex}

\textbf{Komplexer Amplitudendrehzeiger und Ampvektor:}
\begin{align*}
    \underline{\vec{E}}(z,t)
    &= E_0\cdot e^{-\alpha z}\cdot e^{j(\Delta\varphi + \omega t-\beta z)}\cdot\vec{e}_y
    = E_0\cdot e^{-\underline{\gamma}z}\cdot e^{j\omega t}\cdot\vec{e}_y \\
    &= E_0\cdot e^{-\alpha z}\cdot e^{j(\Delta\varphi -\beta z)}\cdot\vec{e}_y
    = E_0\cdot e^{-\underline{\gamma}z}
\end{align*}
\vspace{1ex}

\textbf{Fortpflanzungskonstante}

Dämpfungskonstante $ [\alpha] $= $\frac{\mathtt{Np}}{m}$ \qquad \quad
Phasenkonstante $ [\beta] $= $ \frac{\mathtt{rad}}{m} $
\[\underline{\gamma}=\alpha+j\beta =  \sqrt{j\omega \mu (\kappa + \si{j}\omega \varepsilon)} \quad \left[ \frac{1}{m} \right]\]
 

\textbf{Brechungszahl:}
\[ n = \dfrac{c_0}{c} = \sqrt{\epsilon_r} \geq 1 \]

\textbf{Wellenzahl:}
Im Vakuum: $\beta_{0}=\frac{\omega}{c_{0}}$
\begin{align*}
    \beta & = \frac{2 \pi}{\lambda} = \frac{\omega}{c_0} = \frac{2 \pi f}{v_p} = |\vec{\beta}|                                                                      \\
      	  & = \frac{\omega \cdot n}{c_{0}} = n \cdot \beta_{0}=\sqrt{\mu_{r} \cdot \varepsilon_{r}} \cdot \beta_{0}
\end{align*}

\textbf{Wellenlänge:}
\begin{align*}
    \lambda   & = \dfrac{\lambda_0}{\sqrt{\mu_r \cdot \varepsilon_r}} = \dfrac{c_0}{f \sqrt{\mu_r \cdot \varepsilon_r}} = \dfrac{2 \pi}{\beta} = \dfrac{v_p}{f} = [m] \\
              & = \dfrac{\lambda_0}{n} = \dfrac{2 \pi}{n \cdot k_0}                                             \\
    \lambda_0 & = \dfrac{c_0}{f} = \dfrac{2\pi}{k_0}
\end{align*}

\textbf{Phasengeschwindigkeit:}
$\left(= \sqrt{x^2 + y^2 \dots} \right)$
\[
    v_p = \dfrac{\omega}{\beta} = \frac{c_0}{\sqrt{ \mu_r \varepsilon_r }}  = \frac{1}{\sqrt{ \mu_r \mu_0 \varepsilon_r \varepsilon_0}} \qquad v_{p,\texttt{Medium} \leq c_0}
\]


\subsection{Ausbreitung}
\subsubsection{Allgemein}
\begin{align*}
    \lambda                 & = \dfrac{2\pi}{\beta}         \qquad E_2 = E_1 e^{-\alpha z}                                                                                        \\
    v_p                     & = \lambda\cdot f = \dfrac{\omega}{\beta}                                                                                                    \\
    \alpha                  & = \omega \cdot \sqrt{\dfrac{\mu \varepsilon}{2}\cdot \left(\sqrt{1+\dfrac{\kappa^2}{\omega^2\cdot\varepsilon^2}} - 1\right)}   \\
    \beta                   & = \omega \cdot \sqrt{\dfrac{\mu \varepsilon}{2}\cdot \left(\sqrt{1+\dfrac{\kappa^2}{\omega^2\cdot\varepsilon^2}} + 1\right)} \\
    \Aboxed{\underline{Z}_F & = \dfrac{\underline{E}}{\underline{H}} = \dfrac{\omega \mu}{\beta - j \alpha} = \sqrt{\dfrac{j\omega\mu}{\kappa+j\omega\varepsilon}} \quad\text{komplex, wenn } \alpha \neq 0} 
\end{align*}

\subsubsection{Im leeren Raum(Vakuum)}
\begin{align*}
    \alpha                     & = 0  \qquad \beta = \dfrac{\omega}{c_0}                                                  \\
    \lambda                    & = \dfrac{c_0}{f}                                                       \\
    v_p                        & = c_0        \qquad  \omega = \beta \cdot c_0                                \\
    \Aboxed{\underline{Z}_{F0} & = \sqrt{\dfrac{\mu_0}{\varepsilon_0}} = 120 \pi\Omega\approx377\Omega}
\end{align*}

\subsubsection{Im verlustlosen/idealen Dielektrika}
verlustlos: $\kappa =0$, maximale Wirkleistung

$Z_F$ rein reel $\rightarrow$ ebene Welle
\begin{align*}
    \alpha                  & = 0  \qquad \beta = \dfrac{\omega}{c_0}\sqrt{\mu_r\varepsilon_r}=\omega\sqrt{\mu\varepsilon}=\dfrac{2\pi}{\lambda} \\
    \lambda                 & = \dfrac{c_0}{f}\dfrac{1}{\sqrt{\mu_r\varepsilon_r}}                                             \\
    v_p                     & = \dfrac{c_0}{\sqrt{\mu_r\varepsilon_r}}                                                         \\
    \Aboxed{\underline{Z}_F & = \sqrt{\dfrac{\mu}{\varepsilon}} = Z_{F0} \cdot \sqrt{\dfrac{\mu_r}{\varepsilon_r}}}
\end{align*}

\subsubsection{Im Dielektrika mit geringem Verlust}
geringer Verlust: $\dfrac{\kappa}{\omega\varepsilon} \ll 1$

\begin{align*}
    \alpha                  & \approx\dfrac{\kappa}{2}\cdot\sqrt{\dfrac{\mu}{\varepsilon}} = \frac{\kappa}{2}\cdot Z_{F0}  \qquad \beta \approx\omega\sqrt{\mu\varepsilon}\left(1+\dfrac{1}{8}\cdot\dfrac{\kappa^2}{\omega^2\varepsilon^2}\right) \\
    \lambda                 & = \dfrac{c_0}{f}\cdot\dfrac{1}{\sqrt{\mu_r\varepsilon_r}}\cdot\frac{1}{1+\frac{1}{8}\left(\frac{\kappa}{\omega\varepsilon}\right)^2}                       \\
    v_p                     & = \dfrac{c_0}{\sqrt{\mu_r\varepsilon_r}}\cdot\frac{1}{1+\frac{1}{8}\left(\frac{\kappa}{\omega\varepsilon}\right)^2}                                        \\
    \Aboxed{\underline{Z}_F & = \sqrt{\dfrac{\mu}{\varepsilon}}\left(1-\frac{j\kappa}{\omega\varepsilon}\right)^{-^1/2} = Z_{F0}\left(1+\frac{j\kappa}{2\omega\varepsilon}\right)}
\end{align*}

\subsubsection{Im guten Leiter}
geringer Verlust: $\dfrac{\kappa}{\omega\varepsilon} \gg 1$
\begin{align*}
    \alpha                  & \approx \beta \approx\sqrt{\frac{\omega\mu\kappa}{2}}=\dfrac{1}{\delta}\sim\sqrt{f} \\
    \lambda                 & = 2\pi \sqrt{\dfrac{2}{\omega\mu\kappa}}=2\pi\delta                                 \\
    v_p                     & = \frac{2\pi}{\beta} = \omega\delta                                                 \\
    \Aboxed{\underline{Z}_F & = \sqrt{\dfrac{j\omega\mu}{\kappa}} = \dfrac{1+j}{\kappa\cdot\delta}}
\end{align*}

\subsubsection{Poynting-Vektor}
\begin{tabular}{ll}
	Zeitbereich                                                                        & Frequenzbereich                                                                        \\
	$\vec{S} = \vec{E} \times \vec{H}$                                                 & $\vec{\underline{S}} = \frac{1}{2} (\underline{\vec{E}} \times \underline{\vec{H}}^*)$ \\
	$\vec{S}_{av} = \overline{\vec{S}(t)} = \frac{1}{T} \int_{0}^{T} \vec{S}(t) \,dt $ & $\vec{S}_{av} = \frac{1}{2} \Re{\underline{\vec{E}} \times \underline{\vec{H}}^*}$     \\
	\multicolumn{2}{c}{Leistungsflussdichte, Intensität $\quad S_{av} = |\vec{S}_{av}|$}                                                                                      \\
\end{tabular}
\begin{align*}
	S_{av} & =  \frac{1}{2} \cdot E \cdot H
	=  \frac{1}{2} \cdot \dfrac{E^2}{Z_{F0}}
	=  \frac{1}{2} \cdot H^2 \cdot Z_{F0}
	=  \frac{P}{A_\texttt{Fläche}}                                                                                   \\
	P      & = \iint\vec{S}_{\text{av}}\, d\vec{a}
	= Re\left\{\underline{U}\cdot\underline{I}^*\right\}                                                             \\
	P_1    & = P_0 \cdot e^{-2\alpha z} \qquad P_{\texttt{Leitung}} = \dfrac{1}{2} \dfrac{\hat{U}^2}{\cdot \Re{Z_L}}
\end{align*}

\subsection{Übergang}
\subsubsection{Zwischen Dielektrika mit geringem Verlust}
\begin{align*}
    \lambda_1 & = \dfrac{\lambda_0}{\sqrt{\mu_{r1}\varepsilon_{r1}}}          & \lambda_2 & = \dfrac{\lambda_0}{\sqrt{\mu_{r2}\varepsilon_{r2}}}                                     \\
    \beta_1   & = \dfrac{2\pi}{\lambda_0}\cdot\sqrt{\mu_{r1}\varepsilon_{r1}} & \beta_2   & = \dfrac{2\pi}{\lambda_0}\cdot\sqrt{\mu_{r2}\varepsilon_{r2}}                            \\
    Z_{F1}    & = \dfrac{Z_{F0}}{\sqrt{\mu_{r1}\varepsilon_{r1}}}             & Z_{F2}    & = \dfrac{Z_{F0}}{\sqrt{\mu_{r2}\varepsilon_{r2}}}
\end{align*}

\subsubsection{An Grenzflächen}
\begin{align*}
    S_t &= S_h - S_r \\
    S_t &= S_h  (1 - r_e^2) \cdot \frac{\cos \alpha_h}{\cos \alpha_t}
\end{align*}

\subsection{dÀlembertsche Gleichung (allg.)}
\begin{align*}
    \Delta \vec{E}-\kappa \mu \frac{\partial \vec{E}}{\partial t}-\varepsilon \mu \frac{\partial^{2} \vec{E}}{\partial t^{2}} & = \operatorname{grad} \frac{\rho}{\varepsilon} \\
    \Delta \vec{H}-\kappa \mu \frac{\partial \vec{H}}{\partial t}-\varepsilon \mu \frac{\partial^{2} \vec{H}}{\partial t^{2}} & = 0
\end{align*}

Isolator, ideales Dielektrikum, Nichtleiter $\kappa = 0$
\begin{align*}
    \Delta \vec{E} & =\varepsilon \mu \frac{\partial^{2} \vec{E}}{\partial t^{2}}+\operatorname{grad} \frac{\rho}{\varepsilon} \\
    \Delta \vec{H} & =\varepsilon \mu \frac{\partial^{2} \vec{H}}{\partial t^{2}}
\end{align*}

sehr gute Leiter
\begin{align*}
    \Delta \vec{E} & =\kappa \mu \frac{\partial \vec{E}}{\partial t}+\operatorname{grad} \frac{\rho}{\varepsilon} \\
    \Delta \vec{H} & =\kappa \mu \frac{\partial \vec{H}}{\partial t}
\end{align*}

\subsection{Helmholtz-Gleichungen (Frequenzbereich)}
\begin{align*}
    \Delta \underline{\vec{E}}-\left(\kappa \mu \cdot \mathrm{j} \omega-\varepsilon \mu \cdot \omega^{2}\right) \cdot \underline{\vec{E}} & = \operatorname{grad} \frac{\rho}{\varepsilon} \\
    \Delta \underline{\vec{H}}-\left(\kappa \mu \cdot \mathrm{j} \omega-\varepsilon \mu \cdot \omega^{2}\right) \cdot \underline{\vec{H}} & = 0
\end{align*}

\subsubsection{Zeitbereich}
\begin{align*}
    \Delta \vec{E}-\varepsilon \mu \frac{\partial^{2} \vec{E}}{\partial t^{2}} & =0 \\
    \Delta \vec{H}-\varepsilon \mu \frac{\partial^{2} \vec{H}}{\partial t^{2}} & =0
\end{align*}

\subsubsection{Frequenzbereich (harmonisch)}
\begin{align*}
    \Delta \underline{\vec{E}}+\varepsilon \mu \omega^{2} \cdot \underline{\vec{E}} & =0 \\
    \Delta \underline{\vec{H}}+\varepsilon \mu \omega^{2} \cdot \underline{\vec{H}} & =0
\end{align*}

\textbf{Zeitabhängigkeit harmonisch:}
\begin{align*}
    \Delta \vec{H}   & = (j \omega \mu \kappa - \omega^2 \varepsilon \mu ) \vec{H}                                  \\
    \Delta \vec{E} i & = (j \omega \mu \kappa - \omega^2 \varepsilon \mu ) \vec{E} + grad \frac{ \rho}{\varepsilon}
\end{align*}

\textbf{keine Raumladung $ \rho = 0$}
\begin{align*}
    \Delta \vec{E} & = (j \omega \mu \kappa - \omega^2 \varepsilon \mu ) \vec{E}
\end{align*}

\textbf{Ebene Wellen}
\begin{align*}
    \Delta \vec{E} & = \frac{ \partial \vec{E}}{ \partial z^2} = j \omega \mu ( \kappa + j \omega \varepsilon) \vec{E} \\
    \Delta \vec{H} & = \frac{ \partial \vec{E}}{ \partial z^2} = j \omega \mu ( \kappa + j \omega \varepsilon) \vec{H}
\end{align*}

%%%%%%%%%%%%%%%%%

\subsubsection{Gruppengeschwindigkeit}
\begin{align*}   
    v_g        &= \dfrac{d \omega}{d k} = \dfrac{\textnormal{Wegstück der Wellengruppe}}{\textnormal{Laufzeit der Wellengruppe}}    \\                                                                                                                                                                                     \\
    E(z,t)     & = 2E\cdot\underbrace{\cos(\omega_0t-\beta_0z)}_{\mathclap{\text{Grundfrequenz $\omega$}}}\cdot\underbrace{\cos(\Delta\omega t-\Delta\beta z)}_{\mathclap{\text{Einhüllende $\Delta\omega$}}} \\
    v_p        & = \frac{\omega_0}{\beta_0}                                                                                                                                                                   \\
    v_g        & = \frac{\Delta\omega}{\Delta\beta}
\end{align*}

\subsubsection{Totalrefexion/Grenzwinkel}
\begin{align*}
    \theta_g & = \arcsin \sqrt{ \dfrac{\mu_{r1} \varepsilon_{r1}}{\mu_{r2} \varepsilon_{r2}}} &
\end{align*}

\subsubsection[Brewster-/Polarisationswinkel]{Brewster-/Polarisationswinkel ($r=0$)}
\begin{itemize}
	\item \textbf{Parallele} Polarisation: \quad rechts:
	      $ \mu_{r1}=\mu_{r2} $
	      \begin{align*}
		      \sin\theta_b         & = \sqrt{\frac{\varepsilon_2(\mu_2\varepsilon_1 - \mu_1\varepsilon_2)}{\mu_1(\varepsilon_1^2-\varepsilon_2^2)}} &
		      \Aboxed{\tan\theta_b & = \sqrt{\frac{\varepsilon_2}{\varepsilon_1}} = \frac{n_2}{n_1}}                                                &
	      \end{align*}
	      Brewster-Winkel existiert nur, wenn $ \varepsilon_{r1} \neq \varepsilon_{r2} $.
	\item \textbf{Senkrechte} Polarisation:  \quad rechts: $ \varepsilon_{r1} = \varepsilon_{r2} $
	      \begin{align*}
		      \sin\theta_b & = \sqrt{\frac{\mu_2(\mu_2\varepsilon_1 - \mu_1\varepsilon_2)}{\varepsilon_1(\mu_2^2-\mu_1^2)}} &
		      \tan\theta_b & = \sqrt{\frac{\mu_2}{\mu_1}}                                                                   &
	      \end{align*}
	      Brewster-Winkel existiert nur, wenn $ \mu_{r1} \neq \mu_{r2} $.\\
	      Bei $ \mu_{r1}=\mu_{r2} \rightarrow r \neq 0$ \qquad kein Brewster-Winkel $\theta_b \rightarrow \infty$!
\end{itemize}

\subsection{Stehwellenverhältnis}
\[
    \mathrm{SWR} = \frac{E_{\max}}{E_{\min}}=\frac{H_{\max}}{H_{\min}}=\frac{E_{h}+E_{r}}{E_{h}-E_{r}} = \frac{1+|r|}{1-|r|} \quad 1<s<\infty
\]

\subsection{Polarisation}
Die Polarisation (Ausrichtung) bezieht sich \textbf{immer} auf das $\vec{E}$-Feld.
\begin{itemize}
	\item \textbf{Lineare} Polarisation\\
	Endpunkt des $\vec{E}$-Vektors beschreibt bei fortschreitender Welle eine Gerade (Linie).
		\begin{itemize}
				\item horizontale Polarisation: E-Feld parallel zum Erdboden.
				\item vertikale Polarisation E-Feld senkrecht zum Erdboden.
			\end{itemize}
	\item \textbf{Elliptische} Polarisation\\
	Endpunkt des $\vec{E}$-Vektors beschreibt bei fortschreitender Welle eine Ellipse.
		\begin{itemize}
				\item Zirkulare Polarisation: $|\vec{E}_x| = |\vec{E}_y|$ bei $ \vec{E}_x \perp \vec{E}_y $ ($ \ang{90}$ Phasenverschiebung)
			\end{itemize}
\end{itemize}


\newpage
\subsection[Senkrechter Einfall]{Senkrechter Einfall}
Gilt bei Einfallswinkel $ \theta_h = 0 $.
\input{Figures/Wellen_Senkrechter_Uebergang.tex}

\begin{equation*}
	\setlength{\jot}{6pt}
	\begin{aligned}[t]
        r_e & = \frac{Z_{F2}-Z_{F1}}{Z_{F2} + Z_{F1}} \\
        t_e & = \frac{2 \cdot Z_{F2}}{Z_{F1} + Z_{F2}} \\
		t_e            & = 1+ r_e                                           \\
		E_{t1}         & =E_{t2}                                            \\
		E_t            & = t_e \cdot E_h                                    \\
		E_r            & = r_e \cdot E_h                                    \\
		E_t            & = E_h + E_r                                        \\
		t_e\cdot E_{h} & = E_{h} + r_e\cdot  E_{h}                          \\
		E_t            & = E_h \cdot \frac{2 \cdot Z_{F2}}{Z_{F2} + Z_{F1}} \\
		E_r            & = E_h \cdot \frac{Z_{F2}-Z_{F1}}{Z_{F2}+Z_{F1}}
	\end{aligned}
	\qquad
	\begin{aligned}[t]
        r_m & =-r_e  = \frac{Z_{F1}-Z_{F2}}{Z_{F2} + Z_{F1}}          \\
        t_m & = \frac{2 \cdot Z_{F1}}{Z_{F1} + Z_{F2}} = t_e \cdot \frac{Z_{F1}}{Z_{F2}}   \\
		t_m            & = 1+ r_m                                                      \\
		H_{t1}         & = H_{t2}                                                      \\
		H_t            & = t_m \cdot H_h  = t_e \cdot \tfrac{Z_{F1}}{Z_{F2}} \cdot H_h \\
		H_r            & = r_m \cdot H_h                                               \\
		H_t            & = H_h + H_r                                                   \\
		t_m\cdot H_{h} & = H_{h} + r_m\cdot  H_{h}                                     \\
	\end{aligned}
\end{equation*}

\begin{align*}
	H_t                         & = H_h + H_r                                          \\
	\frac{t\cdot E_{h}}{Z_{F2}} & = \frac{E_{h}}{Z_{F1}} - \frac{r\cdot E_{h}}{Z_{F1}} \\
	\frac{t}{Z_{F2}}            & = \frac{1}{Z_{F1}} - \frac{r}{Z_{F1}}
\end{align*}

\subsubsection[Senkrechter Einfall ideales/verlustl. Dielekt.]{Verlustloses Dielektikum allgemein}
gilt für $ \kappa =0 $, keine Dämpfung.
\[ \text{rein reell: }Z_F=  \sqrt{\frac{\mu_0\mu_r}{\varepsilon_0\varepsilon_r}}  \qquad
	\text{rein imaginär: }\gamma  = j \omega\sqrt{\mu\varepsilon} \]
\begin{align*}
	r & = r_e =\frac{Z_{F2} - Z_{F1}}{Z_{F1} + Z_{F2}} = \frac{\sqrt{\varepsilon_{r1}\mu_{r2}} - \sqrt{\varepsilon_{r2}\mu_{r1}} }{\sqrt{\varepsilon_{r1}\mu_{r2}}+{\sqrt{\varepsilon_{r2}\mu_{r1}}}} \\
	t & = t_e = \frac{2 Z_{F2}}{Z_{F1} + Z_{F2}} = \frac{2\sqrt{\varepsilon_{r1}\mu_{r2}}}{\sqrt{\varepsilon_{r1}\mu_{r2}}+\sqrt{\varepsilon_{r2}\mu_{r1}}}
\end{align*}

\subsubsection{Medium 1 oder 2: Luft}
\begin{equation*}
	\setlength{\jot}{6pt}
	\begin{aligned}[t]
		\Aboxed{ & \mu_{r1} = \varepsilon_{r1} = 1}                                                            \\
		r        & = \frac{\sqrt{\mu_{r2}}-\sqrt{\varepsilon_{r2}}}{{\sqrt{\mu_{r2}}+\sqrt{\varepsilon_{r2}}}} \\
		t        & = \frac{2\sqrt{\mu_{r2}}}{\sqrt{\mu_{r2}}+\sqrt{\varepsilon_{r2}}}
	\end{aligned}
	\qquad
	\begin{aligned}[t]
		\Aboxed{ & \mu_{r2} = \varepsilon_{r2} = 1}                                                            \\
		r        & = \frac{\sqrt{\varepsilon_{r1}}-\sqrt{\mu_{r1}}}{{\sqrt{\varepsilon_{r1}}+\sqrt{\mu_{r1}}}} \\
		t        & = \frac{2\sqrt{\varepsilon_{r1}}}{\sqrt{\mu_{r1}}+\sqrt{\varepsilon_{r1}}}
	\end{aligned}
\end{equation*}

\subsubsection{beide Medien: nicht magnetisch}
\begin{align*}
	r & = \frac{\sqrt{\varepsilon_{r1}}-\sqrt{\varepsilon_{r2}}}{{\sqrt{\varepsilon_{r1}}+\sqrt{\varepsilon_{r2}}}} &
	t & = \frac{2}{1+\sqrt{\frac{\varepsilon_{r2}}{\varepsilon_{r1}}}}                                              &
\end{align*}

\subsubsection{Medium 2: idealer Leiter}
$\vec{E}=0$ im idealen Leiter $\rightarrow$ \textbf{Stehende} Welle!, vollständige Reflexion.
\begin{equation*}
	Z_{F2}       = 0 \qquad
	r            = -1 \qquad
	t            = 0 \qquad
	\vec{S}_{\text{av}} = 0
\end{equation*}

$E$ und $H$: zeitlich sowie örtlich zueinander um \ang{90} phasenverschoben.
\begin{flalign*}
	 & \underline{E}_{1x}          = -2j\cdot E_{h1}\cdot \sin(\beta_1 z)\qquad
	\underline{H}_{1y}          = 2\cdot \tfrac{E_{h1}}{Z_{F1}}\cdot \cos(\beta_1 z)  \\
	 & E_{1x} (z,t)=2E_{h1} \cdot \sin(\beta_1z)\cdot \sin(\omega t)                  \\
	 & H_{1y}(z,t) = 2 \tfrac{E_{h1}}{Z_{F1}}\cdot \cos(\beta_1z)\cdot \cos(\omega t)
\end{flalign*}

Annahme: Grenzfläche bei $z=0$.
\begin{align*}
	\Aboxed{ & \text{$H_{\text{max}}$, $E_{\text{min}}$ bei } z = - n \cdot \lambda/_2} & \\
	\Aboxed{ & \text{$H_{min}$, $E_{max}$ bei } z=- (2n-1) \cdot \lambda/_4}            &
\end{align*}

\newpage
\subsection{Senkrechte (E-Feld) Polarisation (H-Feld parallel)}
\input{Figures/SchraegerUebergang_senk_Polarisation.tex}

\begin{flalign*}
	Z_{F(n)}                & = Z_{F0}\cdot\sqrt{\frac{\mu_{r(n)}}{\varepsilon_{r(n)}}}&
	\frac{Z_{F1}}{Z_{F2}} & = \frac{\sqrt{\mu_{r 1}\varepsilon_{r2}}}{\sqrt{\mu_{r 2}\varepsilon_{r1}}}&
\end{flalign*}

\textbf{Brechungsgesetz}: \qquad  mit $ \theta_h = \theta_r\ $
\begin{flalign*}
	\Aboxed{\frac{\sin\theta_t}{\sin\theta_h} & = \sqrt{\frac{\mu_{r 1}\varepsilon_{r1}}{\mu_{r 2}\varepsilon_{r2}}}} = \frac{\lambda_2}{\lambda_1}= \frac{\beta_1}{\beta_2}= \frac{n_1}{n_2} &
\end{flalign*}

\textbf{Fresnelsche Formeln ohne $\mu_r$}
\begin{flalign*}
		    r_s    & =  r_{es} = r_{ms} =                                                                                                                                            \\
		& = \frac{Z_{F2} \cdot \cos \theta_h-Z_{F1} \cdot \cos \theta_t}{Z_{F2} \cdot \cos \theta_h+Z_{F1} \cdot \cos \theta_t}                                           
%		& = \frac{\cos\theta_h-\sqrt{^{\varepsilon_{r2}}/_{\varepsilon_{r1}}-\sin^2\theta_h}}{\cos\theta_h+\sqrt{^{\varepsilon_{r2}}/_{\varepsilon_{r1}}-\sin^2\theta_h}} \\
		= \frac{\sqrt{\varepsilon_{r1}}\cdot\cos\theta_h - \sqrt{\varepsilon_{r2}}\cdot\cos\theta_t}{\sqrt{\varepsilon_{r2}}\cdot\cos\theta_t + \sqrt{\varepsilon_{r1}}\cos\theta_h} \\
		& = \frac{\cos \theta_h-\sqrt{\frac{\varepsilon_{r 2}}{\varepsilon_{r 1}}-\sin ^2 \theta_h}}{\cos \theta_h+\sqrt{\frac{\varepsilon_{r 2}}{\varepsilon_{r 1}}-\sin ^2 \theta_h}}
		\\
		t_{es} & =\frac{2 \cdot	 Z_{F2} \cdot \cos \theta_h}{Z_{F2} \cdot \cos \theta_h+Z_{F1} \cdot \cos \theta_t}                                                               = \frac{2\cdot\sqrt{\varepsilon_{r1}}\cdot\cos\theta_h}{\sqrt{\varepsilon_{r2}}\cdot\cos\theta_t + \sqrt{\varepsilon_{r1}}\cdot\cos\theta_h}
		\\
		& = 1+r_s = \frac{2 \cos \theta_h}{\cos \theta_h+\sqrt{\frac{\varepsilon_{r 2}}{\varepsilon_{r 1}}-\sin ^2 \theta_h}} 
		\\
		t_{ms} & = \frac{2 Z_{F1} \cdot \cos \theta_h}{Z_{F2} \cdot \cos \theta_h+Z_{F1} \cdot \cos \theta_t} = \frac{2 \sqrt{\frac{\varepsilon_{r 2}}{\varepsilon_{r 1}} \cos \theta_h}}{\cos \theta_h+\sqrt{\frac{\varepsilon_{r 2}}{\varepsilon_{r 1}}-\sin ^2 \theta_h}}
		\\                                                                                                          \\
		& = \frac{Z_{F1}}{Z_{F2}}\cdot t_{es} = \sqrt{\frac{\varepsilon_{r2}}{\varepsilon_{r1}}}\cdot t_{es}
\end{flalign*} 

\textbf{Fresnelsche Formeln mit $\mu_r$}
\begin{equation*}
	\setlength{\jot}{10pt}
	\begin{aligned}
		r_s    & =  r_{es} = r_{ms} =                                                                                                                                            \\
		& = \frac{Z_{F2} \cdot \cos \theta_h-Z_{F1} \cdot \cos \theta_t}{Z_{F2} \cdot \cos \theta_h+Z_{F1} \cdot \cos \theta_t}                                           \\
        & =\frac{\cos \theta_h-\sqrt{\frac{\mu_{r 1} \varepsilon_{r 2}}{\mu_{r 2} \varepsilon_{r 1}}-\frac{\mu_{r 1}{ }^2}{\mu_{r 2}^2} \sin ^2 \theta_h}}{\cos \theta_h+\sqrt{\frac{\mu_{r 1} \varepsilon_{r 2}}{\mu_{r 2} \varepsilon_{r 1}}-\frac{\mu_{r 1}{ }^2}{\mu_{r 2}^2} \sin ^2 \theta_h}} \\
		t_{es} & =\frac{2 \cdot	 Z_{F2} \cdot \cos \theta_h}{Z_{F2} \cdot \cos \theta_h+Z_{F1} \cdot \cos \theta_t}                                                              \\
        & = 1+r_s =\frac{2 \cos \theta_h}{\cos \theta_h+\sqrt{\frac{\mu_{r 1} \varepsilon_{r 2}}{\mu_{r 2} \varepsilon_{r 1}}-\frac{\mu_{r 1}{ }^2}{\mu_{r 2}{ }^2} \sin ^2 \theta_h}} \\
		t_{ms} & = \frac{2 Z_{F1} \cdot \cos \theta_h}{Z_{F2} \cdot \cos \theta_h+Z_{F1} \cdot \cos \theta_t}                                                              \\
		& =\frac{2 \sqrt{\frac{\mu_{r 1} \varepsilon_{r 2}}{\mu_{r 2} \varepsilon_{r 1}}} \cos \theta_h}{\cos \theta_h+\sqrt{\frac{\mu_{r 1} \varepsilon_{r 2}}{\mu_{r 2} \varepsilon_{r 1}}-\frac{\mu_{r 1}{ }^2}{\mu_{r 2}{ }^2} \sin ^2 \theta_h}}
		\\
		& = \frac{Z_{F1}}{Z_{F2}}\cdot t_{es} =\sqrt{\frac{\mu_{r 1} \varepsilon_{r 2}}{\mu_{r 2} \varepsilon_{r 1}}} t_{e s}                                                                                                                   
	\end{aligned}
\end{equation*}


\subsection{Senkrechte (H-Feld) Polarisation (E-Feld parallel) }
\begin{center}
\input{Figures/SchraegerUebergang_para_Polaristion.tex}
\end{center}

\textbf{Fresnelsche Formeln ohne $\mu_r$}
\begin{equation*}
		\setlength{\jot}{10pt}
	\begin{aligned}
		r_{ep}    & =  r_{mp} = r_{p} 
		\\
		& = \frac{Z_{F1} \cdot \cos \theta_h-Z_{F2} \cdot \cos \theta_t}{Z_{F1} \cdot \cos \theta_h+Z_{F2} \cdot \cos \theta_t}
		\\
		& =\frac{\cos \theta_h-\sqrt{\frac{\varepsilon_{r 1}}{\varepsilon_{r 2}}-\frac{\varepsilon_{r 1}{ }^2}{\varepsilon_{r 2}{ }^2} \sin ^2 \theta_h}}{\cos \theta_h+\sqrt{\frac{\varepsilon_{r 1}}{\varepsilon_{r 2}}-\frac{\varepsilon_{r 1}{ }^2}{\varepsilon_{r 2}{ }^2} \sin ^2 \theta_h}} \\
		t_{ep} & =  \frac{2 \cdot Z_{F2}   \cdot  \cos \theta_h}{Z_{F1} \cdot \cos \theta_h+Z_{F2} \cdot \cos \theta_t}                                                                                                                           = (1-r_p) \cdot \dfrac{\cos \theta_h}{\cos \theta_t}                                                                                                                                                                        \\
		& = \frac{2 \sqrt{\frac{\varepsilon_{r 1}}{\varepsilon_{r 2}}} \cos \theta_h}{\cos \theta_h+\sqrt{\frac{\varepsilon_{r 1}}{\varepsilon_{r 2}}-\frac{\varepsilon_{r 1}^2}{\varepsilon_{r 2}{ }^2} \sin ^2 \theta_h}} \\
		& = \frac{Z_{F2}}{Z_{F1}}\cdot t_{mp} = \sqrt{\frac{\varepsilon_{r1}}{\varepsilon_{r2}}}\cdot t_{mp} \\
		t_{mp} & = \frac{2 \cdot  Z_{F1}\cdot \cos \theta_h}{Z_{F1} \cdot \cos \theta_h+Z_{F2} \cdot \cos \theta_t}                                                                                                                          
		= 1+r_p                                                                                                                                                                                                                
	\end{aligned}
\end{equation*}

\textbf{Fresnelsche Formeln mit $\mu_r$}
\begin{equation*}
	\setlength{\jot}{10pt}
	\begin{aligned}
		r_{ep}    & =  r_{mp} = r_{p} 
		\\
		& = \frac{Z_{F1} \cdot \cos \theta_h-Z_{F2} \cdot \cos \theta_t}{Z_{F1} \cdot \cos \theta_h+Z_{F2} \cdot \cos \theta_t}
		\\
		& =\frac{\cos \theta_h-\sqrt{\frac{\mu_{r 2} \varepsilon_{r 1}}{\mu_{r 1} \varepsilon_{r 2}}-\frac{\varepsilon_{r 1}{ }^2}{\varepsilon_{r 2}{ }^2} \sin ^2 \theta_h}}{\cos \theta_h+\sqrt{\frac{\mu_{r 2} \varepsilon_{r 1}}{\mu_{r 1} \varepsilon_{r 2}}-\frac{\varepsilon_{r 1}{ }^2}{\varepsilon_{r 2}{} ^2} \sin ^2 \theta_h}} 
        \\
		t_{ep} & =  \frac{2 \cdot Z_{F2}   \cdot  \cos \theta_h}{Z_{F1} \cdot \cos \theta_h+Z_{F2} \cdot \cos \theta_t}
        \\                                                                                                                                                                                                                                                                                         
        &=\frac{2 \sqrt{\frac{\mu_{r 2} \varepsilon_{r 1}}{\mu_{r 1} \varepsilon_{r 2}}} \cos \theta_h}{\cos \theta_h+\sqrt{\frac{\mu_{r 2} \varepsilon_{r 1}}{\mu_{r 1} \varepsilon_{r 2}}-\frac{\varepsilon_{r 1}{ }^2}{\varepsilon_{r 2}{ }^2} \sin ^2 \theta_h}} 
        \\
		& = \frac{Z_{F2}}{Z_{F1}}\cdot t_{mp} = \sqrt{\frac{\mu_{r2}\varepsilon_{r1}}{\mu_{r1}\varepsilon_{r2}}}\cdot t_{mp} 
		\\
		t_{mp} & = \frac{2 \cdot  Z_{F1}\cdot \cos \theta_h}{Z_{F1} \cdot \cos \theta_h+Z_{F2} \cdot \cos \theta_t}                    
        \\
		& = 1+r_p =\frac{2 \cos \theta_h}{\cos \theta_h+\sqrt{\frac{\mu_{r 2} \varepsilon_{r 1}}{\mu_{r 1} \varepsilon_{r 2}}-\frac{\varepsilon_{r 1}{ }^2}{\varepsilon_{r 2}{ }^2} \sin ^2 \theta_h}}  		
	\end{aligned}
\end{equation*}