\section{Leitungen}
\subsection{Allgemeine Lösung Leitungsgleichung}
\begin{align*}
    \underline{U}(\ell)  & = U_h e^{\underline{\gamma} \ell} + U_r e^{-\underline{\gamma} \ell} \\
    \underline{I}(\ell)  & = I_h e^{\underline{\gamma} \ell} + I_r e^{-\underline{\gamma} \ell} \\
    \underline{Z}_L     & = \frac{U_h}{I_h} = \sqrt{ \frac{R + j \omega L}{G + j \omega C}}                                                                                   \\
    \underline{\gamma}  & = \alpha + j\beta = \sqrt{(R+j\omega L)\cdot(G+j\omega C)}                                                                                                            \\
    \lambda             & = \frac{2 \pi}{\beta} \qquad v_p = \frac{\omega}{\beta} \\
    l_\texttt{elek.}    & = \beta \cdot l                                                                                                                                     \\
    \alpha              & = \frac{C R+L G}{2 \sqrt{L C}}=\frac{R}{2 Z_L}+\frac{Z_L G}{2}=\alpha_L+\alpha_D \\
    \beta               & = (j) \omega \sqrt{L C}
\end{align*}
\subsubsection{Reflexionsfaktor}
\begin{align*}
    \underline{r}(\ell)     & = \frac{\underline{U}_r(\ell)}{\underline{U}_h(\ell)} = -\frac{\underline{I}_r(\ell)}{\underline{I}_h(\ell)} = \frac{\underline{U}_r(\ell=0) \cdot \mathrm{e}^{-\mathrm{j} \beta \ell}}{\underline{U}_h(\ell=0) \cdot \mathrm{e}^{+\mathrm{j} \beta \ell}} \\
                            & = \underline{r}(\ell=0) \cdot \mathrm{e}^{-2 \underline{\gamma} \ell}=\underline{r}(\ell=0) \cdot \mathrm{e}^{-2 \alpha \ell} \cdot \mathrm{e}^{-\mathrm{j} 2 \beta \ell}\\
                            & = \frac{\frac{Z_L}{\underline{Z}(\ell)}-1}{\frac{Z_L}{\underline{Z}(\ell)}+1} \\
    \underline{r}           & = \frac{\underline{z}_n-1}{\underline{z}_n+1} \qquad              \underline{z}_n=\frac{\underline{Z}_n}{Z_L}\\
    \underline{r}(\ell = 0) & = \frac{\underline{U}_A-\underline{I}_A Z_L}{\underline{U}_A + \underline{I}_A Z_L}=\frac{\underline{Z}_A-Z_L}{\underline{Z}_A+Z_L}=\frac{\frac{\underline{Z}_A}{Z_L}-1}{\frac{\underline{Z}_A}{Z_L}+1}\\
                    \alpha  & = -\frac{\ln(\frac{r_A}{r_E})}{2l} [\si{Np/m}]  \qquad  \beta = \dfrac{\phi_E -\phi_A}{2l} \left[\cdot\frac{\pi}{180^\circ}\right] [\si{rad/m}]
\end{align*}

\subsubsection{Verlustlose Übertragungsleitung }
\begin{align*}
    \beta              & = \omega\sqrt{LC} = \frac{2 \pi}{\lambda} = \frac{\omega}{v_p}\qquad \alpha = 0 \\
    Z_L                & =\frac{U_h}{U_r}       = \sqrt{\frac{L}{C}}                                                                          \\
    v_p                & = \frac{\omega}{\beta} = \frac{1}{\sqrt{LC}}= \frac{1}{\sqrt{\mu\varepsilon}}= \frac{c_0}{\sqrt{\mu_r\varepsilon_r}} \\
    \lambda            & = \frac{2\pi}{\beta}=\frac{1}{f\sqrt{LC}}= \frac{v_p}{f}= \frac{c_0}{f\sqrt{\mu_r\varepsilon_r}}
\end{align*}


\subsection{Übertragungsleitung mit Last}
\begin{center}
\resizebox{\columnwidth}{!}{
        \begin{circuitikz}%[american voltages]
            %Schaltbild
            \draw(0,0)
            to[V,v=$u_G(t)$](0,2)                               %Spannungsquelle
            to[R=$Z_g$](3,2)                                    %Quelleninnenwiderstand
            to[short,o-o](7,2)                                  %Leitung mit Knoten
            to[short](8 ,2)
            to[R=$Z_A$](8,0)                                    %Lastwiderstand
            to[short](7,0)                                      
            to[short,o-o](3,0)                                  %Leitung mit Knoten
            to[short](0,0);   

            %Knoten + Leitung Beschreibung
            \draw(3,2) node[above] {Eingang/Anfang};
            \draw(7,2) node[above] {Ausgang/Ende};
            \draw[decoration={brace},decorate]
                 (3,2.6) -- node[above=6pt] {$\underline{Z}_L$} (7,2.6);
            
            %linke gestrichelte linie
            \draw[dotted](3,0)--(3,-0.5) node[left]{$l=-d$};
            \draw[dotted](3,-0.5)--(3,-1) node[left]{$z=d$};
            \draw[dotted](3,-1)--(3,-1.5);

            %Pfeil in richtung l
            \draw[-latex](3,-0.5) -- (7,-0.5);
            \node at (4,-0.5)[above]{positiv $l$};
            
            %rechte gestrichelte Linue
            \draw[dotted](7,0)--(7,-0.5) node[right]{$l=0$};          
            \draw[dotted](7,-0.5)--(7,-1) node[right]{$z=0$};
            \draw[dotted](7,-1)--(7,-1.5);

            %pfeil in richtung z
            \draw[latex-](3,-1) -- (7,-1);
            \node at (6,-1)[above]{positiv $z$};

            %Pfeil in hinlaufende richtung
            \draw[-latex](3,1.25) -- (6.5,1.25);
            \node at (4.5,1.25)[above]{hinlaufende Welle};

            %Pfeil in rücklaufende richtung
            \draw[latex-](3.5,0.5) -- (7,0.5);
            \node at (5.5,0.5)[above]{rücklaufende Welle};

            %Pfeil Eingangs Widerstand
            \draw[-latex](1,0.5) -- (2.8,0.5);
            \node at (2,0.5)[above]{$\underline{Z}_E$};
        \end{circuitikz}
}
\end{center}



% \subsubsection{Vorgehen Eingangswiderstand}
% Wenn mit Smithdiagramm gearbeitet wird liefert dieses Schritte \ref{Ref L_anfang} und \ref{Bestimmen Z_E}
% \begin{enumerate}
%     \item Lastimpedanz
%           \[ \underline{Z}_A = \dfrac{1}{\frac{1}{R_A} + j \omega C_A} \]
%     \item Reflexion am Leitungsende
%           \[ \underline{r}(z=0) = \dfrac{Z_A - \underline{Z}_L}{Z_A + \underline{Z}_L} \]
%     \item Reflexion am Leitungsanfang \label{Ref L_anfang}
%           \[ \underline{r}(\ell = -d) =  \underline{r}_A \cdot e^{-j 2 \beta d}\]
%     \item Bestimmung der Impedanz \label{Bestimmen Z_E}
%           \[ \underline{Z}_E = \underline{Z}_L \cdot \dfrac{1 + \underline{r}_E}{1 - \underline{r}_E}\]
%     \item Eingangswiderstand
%           \[ \underline{Z}_E = \dfrac{1}{\frac{1}{\underline{Z}_E} + j \omega C_E}\]
% \end{enumerate}

\subsubsection{Fall: Angepasste Leitung}
\begin{align*}
    Z_A          & = Z_L = Z(\ell)                              \\
    r_A          & = 0\qquad\rightarrow\text{reflexionsfrei} \\
    \mathrm{SWR} & = 1                                       \\
    U(\ell)         & = U_h\cdot e ^{j\beta \ell}                  \\
    I(\ell)         & = I_h \cdot e^{j\beta \ell}                  \\
                 & = \frac{U_h}{Z_L}\cdot e^{j\beta \ell}
\end{align*}

\subsubsection{Fall: Kurzgeschlossene Leitung}
\begin{align*}
    Z_A          & = 0                                                                                         \\
    Z(\ell)         & = j Z_L\cdot\tan(\beta \ell)        \qquad\rightarrow\text{rein imaginär}                      \\
    r_A          & = -1                                                                                        \\
    \mathrm{SWR} & = \infty                                                                                    \\
    U(\ell)         & = U_h\cdot 2j\sin(\beta \ell)    \qquad\rightarrow U(\ell=0)=0                                    \\
    \hat{U}_E    & = \hat{U}_{G}\cdot\frac{\underline{Z}_E}{\underline{Z}_{G}+\underline{Z}_E} \\
    I(\ell)         & = U_h\cdot 2\cos(\beta \ell)    \qquad\rightarrow I(\ell=0)=I_A=\frac{2U_h}{Z_L}
\end{align*}

\subsubsection{Fall: Leerlaufende Leitung}
\begin{align*}
    Z_A          & = \infty                                                                         \\
    Z(\ell)         & = -jZ_L\cdot \frac{1}{\tan (\beta \ell)} \qquad\rightarrow\text{rein imaginär}                 \\
    r_A          & = 1                                                                              \\
    \mathrm{SWR} & = \infty                                                                         \\
    U(\ell)         & = U_h\cdot 2\cos(\beta \ell) \qquad\rightarrow U(\ell=0)=0                             \\
    I(\ell)         & = U_h\cdot 2j\sin(\beta \ell) \qquad\rightarrow I(\ell=0)=I_A = \frac{2\cdot U_h}{Z_L}
\end{align*}

\subsubsection{Fall: Beliebiger Abschluss}
\begin{align*}
    \underline{U}(\ell) &= \underline{U}_A\left[\cos (\beta \ell)+\mathrm{j} \frac{\underline{Z}_L}{\underline{Z}_A} \sin (\beta \ell)\right] \\
    \underline{I}(\ell) &= \underline{I}_A\left[\cos (\beta \ell)+\mathrm{j} \frac{\underline{Z}_A}{\underline{Z}_L} \sin (\beta \ell)\right] \\
    \underline{Z}_E     &= \frac{U(-\ell)}{\underline{I}(-\ell)}= \underline{Z}_A \cdot \frac{1+j \frac{Z_W}{\underline{Z}_A} \tan \beta \ell}{1+j \frac{\underline{Z}_A}{Z_W} \tan \beta \ell} 
\end{align*}
Besonderheit bei $L=\frac{\lambda}{4}$ :
$ \underline{Z}_E=\frac{Z_{W}^2}{\underline{Z}_A}$

\subsubsection{Fall: Ohm'sch abgeschlossene Leitung}
\begin{align*}
                          & r_A = \texttt{reell} \\
    \underline{R_A > Z_L} & \rightarrow\theta_r = 0 \rightarrow r_A \texttt{ ist negativ} \\
                          & \rightarrow \ell_\texttt{max}=\frac{\lambda}{2}\cdot n \\
    \underline{R_A < Z_L} & \rightarrow\theta_r = \pi                           \\
                          & \rightarrow \ell_\texttt{min}=\frac{\lambda}{2}\cdot n
\end{align*}

\subsubsection{Stehwellenverhältnis/ Anpassungsfaktor}
siehe auch Kap. \ref{sec:Smith_All}
\begin{align*}
    \mathrm{SWR}      & = \frac{U_\text{max}}{U_\text{min}} = \frac{I_\text{max}}{I_\text{min}} = \frac{1+|r(\ell)|}{1-|r(\ell)|} = \frac{|U_h|+|U_r|}{|U_h|-|U_r|} \\
    m                 & = \frac{1}{\mathrm{SWR}} = \frac{1 - |r(\ell)|}{1 + |r(\ell)|}
\end{align*}


\subsubsection{vernachlässigbarer Widerstandsbelag}
\includegraphics[width=\columnwidth]{Figures/vernachlaessigbarerWiderstandsbelag.png}

\subsubsection{vernachlässigbarer Leitwertbelag}
\includegraphics[width=\columnwidth]{Figures/vernachlaessigbarerLeiterwertbelag.png}


\subsubsection{Leistung}
\begin{align*}
    P_{A}            & = P_{H}-P_{R}                                                                                                 \\
                     & = \frac{1}{2} \cdot \frac{\hat{U}_{h}^{2}}{Re\{Z_{L}\}}-\frac{1}{2} \cdot \frac{\hat{U}_{r}^{2}}{Re\{Z_{L}\}} \\
                     & =\frac{1}{2} \cdot \frac{\hat{U}_{h}^{2}}{Re\{Z_{L}\}} \cdot\left(1-r^{2}\right)                              \\
                     & = P_{\max} \cdot\left(1-r^{2}\right)                                                                          \\
                     & = \underline{U}_A\cdot\underline{I}_A^*                                                                       \\
    P_V              & = P_q -P_A                                                                                                    \\
    \underline{I}(\ell) & = \hat{I}\cdot e^{-\alpha \ell}\angle \beta \ell
\end{align*}
\subsubsection{Gleichspannungswert (=Endwert)}
\begin{align*}
    U_A & = U_q\cdot\frac{R_A}{R_i+R_A}
\end{align*}

\subsubsection{Position von Extrema}
\begin{align*}
    \lambda_\texttt{min/max}    & = \frac{c_0}{f_\texttt{min/max}\sqrt{\mu_{r1}\varepsilon_{r1}}}\\
    z_\texttt{min}              & =\frac{-n\cdot\lambda_\texttt{min}}{2}                                        \qquad\rightarrow n = -\frac{2z}{\lambda_\texttt{min}}                            \\
    z_\texttt{max}              & =\frac{-(2n+1)\lambda_\texttt{max}}{4}                                        \qquad\rightarrow n = -\frac{4z+\lambda_\texttt{max}}{2\cdot\lambda_\texttt{max}} \\
    z                           & = \frac{\lambda_\texttt{min}\cdot\lambda_\texttt{max}}{4(\lambda_\texttt{min}-\lambda_\texttt{max})}
\end{align*}

\subsection{Mehrfachreflexionen bei fehlender Anpassung}
\begin{center}
    \resizebox{\columnwidth}{!}{
    \begin{tikzpicture}
        %Linien
        \draw[-Latex] (1,1) -- (1,0) node [below] {$t$};
        \draw[-,line width=1pt] (1,1) -- (1,6);
        \draw[-,line width=1pt] (5,0) -- (5,6);

        %Pfeile mit Bezeichnungen
        \draw[-Latex] (3.5,6.5) -- (5,6.5)node[right]{$z$};

        \draw[-Latex] (1,6) -- (5,5) node[right]{$t_d$} node[midway, above]{$u_{1h}$};
        %\draw[-] (1,6) -- (3,5.5) node[above]{$U_{1h}$};

        \draw[-Latex] (5,5) -- (1,4)node[left]{$2\cdot t_d$} node[midway, above]{$u_{1r}$};
        %\draw[-] (5,5) -- (3,4.5) node[above]{$U_{1r}$};

        \draw[-Latex] (1,4) -- (5,3)node[right]{$3\cdot t_d$} node[midway, above]{$u_{2h}$};
        %\draw[-] (1,4) -- (3,3.5) node[above]{$U_{2h}$};

        \draw[-Latex] (5,3) -- (1,2)node[left]{$4\cdot t_d$} node[midway, above]{$u_{2r}$};
        %\draw[-] (5,3) -- (3,2.5) node[above]{$U_{2r}$};

        \draw[-Latex] (1,2) -- (5,1)node[right]{$5\cdot t_d$} node[midway, above]{$u_{3h}$};
        %\draw[-] (1,2) -- (3,1.5) node[above]{$U_{3h}$};


        \draw[dotted ] (5,1) -- (3,0.5);

        %Klammern mit Bezeichnungen
        \draw [black,
            decorate,
            decoration = {brace,
                    raise=5pt,
                    amplitude=5pt}] (5,5.8) --  (5,5.2);
        \node at (5.5,5.5)[right]{$u_A = 0$};

        \draw [black,
            decorate,
            decoration = {brace,
                    raise=5pt,
                    amplitude=5pt}] (1,4.2) --  (1,5.8);
        \node at(0.5,5)[left]{$u_E = u_{1h}$};

        \draw [black,
            decorate,
            decoration = {brace,
                    raise=5pt,
                    amplitude=5pt}] (5,4.8) --  (5,3.2);
        \node at (5.5,4)[right]{$u_A = u_{1h}(1+\underline{r}_A)$};

        \draw [black,
            decorate,
            decoration = {brace,
                    raise=5pt,
                    amplitude=5pt}] (1,2.2) --  (1,3.8);
        \node at (0.5,3)[left]{$u_E = u_{1h}$};
        \node at (0.5,2.5)[left]{$+(1+\underline{r}_I)u_{1r}$};

        \draw [black,
            decorate,
            decoration = {brace,
                    raise=5pt,
                    amplitude=5pt}] (5,2.8) --  (5,1.2);
        \node at (5.5,2)[right]{$u_A = u_{1h}(1+\underline{r}_A)$};
        \node at (5.5,1.5)[right]{$+u_{2h}(1+\underline{r}_A)$};


        \draw [black,
            decorate,
            decoration = {brace,
                    raise=5pt,
                    amplitude=5pt}] (1,0.2) --  (1,1.8);
        \node at (0.5,1.5)[left]{$u_E = u_{1h}$};
        \node at (0.5,1)[left]{$+(1+\underline{r}_I)u_{1r}$};
        \node at (0.5,0.5)[left]{$+(1+\underline{r}_I)u_{2r}$};
    \end{tikzpicture}
    }
\end{center}

\begin{align*}
    %u_{1h} & = u_G\cdot\frac{ Z_L}{R_I + Z_L}            \\
    u_{1r} & = r_A\cdot u_{1h}                                \\
    u_{2h} & = r_I\cdot u_{1r} = r_I\cdot r_A\cdot u_{1h}     \\
    u_{2r} & = r_A\cdot u_{2h} = r_I\cdot r_A^2\cdot u_{1h}   \\
    u_{3h} & = r_I\cdot u_{2r} = r_I^2\cdot r_A^2\cdot u_{1h}
\end{align*}
\input{Figures/Leitungen_Mehrfach_Reflexion_Circuit.tex}
\begin{align*}
     & \text{Reflexionsfaktor Leitungsanfang: } & \underline{r}_I & = \frac{R_I - Z_L}{R_I + Z_L}                 \\
     & \text{Reflexionsfaktor Leitungsende: }   & \underline{r}_A & = \frac{R_A - Z_L}{R_A + Z_L}                 \\
     & \text{Hinlaufende Welle}                 & u_{1h}          & = \hat{u}_G \cdot\frac{Z_L}{Z_L+R_I}          \\
     & \text{Signallaufzeit: }                  & t_d             & = \frac{l}{c_0}\cdot\sqrt{\mu_r\varepsilon_r} \\
     &                                          &                 & = \frac{l}{v_p}
\end{align*}
\subsection{Kettenmatrix einer Leitung}
\[
    A =
    \left[ {\begin{array}{cc}
                    \cosh(\gamma l)               & Z_L \sinh(\gamma l) \\
                    \frac{1}{Z_L} \sinh(\gamma l) & \cosh(\gamma l)     \\
                \end{array} } \right]
\]

\subsection{Leitungsparameter}

{\small\[
        \sigma = \text{Leitwert des Dielektr.} \qquad \sigma_c = \text{Leitwert des Leiters}
    \]}

\subsubsection{Parallele Platten}
{\small\[
        w  = \text{Platten Breite} \qquad d  = \text{Abstand zw. Platten}
    \]}

Für Sinus-Anregung:
\begin{align*}
    I & = \frac{U}{Z_L} = \underbrace{\frac{U_0}{Z_L}}_{I_0}\cdot e^{-j\beta z\cdot e^{j\omega t}}                         \\
    U & = \int \vec{E} d\vec{s} \stackrel{w\gg d}{=} E\cdot d \rightarrow E = \frac{U_0}{d}\cdot^{-j\beta z}\cdot\vec{e}_x \\
    I & = \oint \vec{H} d\vec{s} =  H\cdot w \rightarrow H = \frac{I_0}{w}\cdot^{-j\beta z}\cdot\vec{e}_y                  % \\
    % \vec{E}(r, z) & = \frac{I}{2\pi r}\cdot Z_F\cdot e^{-j\beta z} \cdot\vec{e}_r                                                      \\
    %               & = \frac{\hat{U}}{r \cdot\ln{(^{2b}/_{2a})}}\cdot e^{-j\beta z}\cdot\vec{e}_r
\end{align*}

\input{Figures/Leitungen_Parallele_Platten.tex}
{\renewcommand*{\arraystretch}{0.2}
    \begin{tabularx}{0.5\columnwidth}{|X|}
        \hline
        \[R=\frac{2}{w\delta\sigma}\] \\
        \hline
        \[L=\frac{\mu d}{w}\]         \\
        \hline
        \[G=\frac{\sigma w}{d}\]      \\
        \hline
        \[C=\frac{w\varepsilon}{d}\]  \\
        \hline
    \end{tabularx}
}

\subsubsection{Doppelleitung:}
{\small\[
        a = \text{Leiter Radius} \qquad d = \text{Abstand zw. den Leitern}\\
    \]}
{\small\[
        \text{cosh am TR: MENU $\rightarrow$ 1; OPTN $\rightarrow$ 1 $\rightarrow$ 5}\\
    \]}
\input{Figures/Leitungen_Doppelleitung.tex}
{\renewcommand*{\arraystretch}{0.2}
    \begin{tabularx}{0.5\columnwidth}{|X|}
        \hline
        \[R  = \frac{1}{\pi a\delta\sigma_c}\]              \\
        \hline
        \[L = \frac{\mu}{\pi} \cosh^{-1}\frac{d}{2a}\]      \\
        \hline
        \[G = \frac{\pi\sigma}{\cosh^{-1}(^d/_{2a})}\]      \\
        \hline
        \[C = \frac{\pi\varepsilon}{\cosh^{-1}(^d/_{2a})}\] \\
        \hline
    \end{tabularx}}

\subsubsection{Koaxial Leitung}
\begin{align*}
    a                     & = \text{innen Radius} \qquad b = \text{außen Radius} \\
    r                     &< a \text{ und } r > b \Rightarrow \vec{E}, \vec{H} = 0 \\
    \vec{H}(r, z)         & = \frac{\hat{I}}{2\pi r}\cdot e^{-j\beta z}\cdot\vec{e}_\varphi                   \\
    \vec{E}(r, z)         & = \frac{\hat{I}}{2\pi r}\cdot Z_{L}\cdot e^{-j\beta z} \cdot\vec{e}_r = \frac{\hat{U}}{r \cdot\ln{(^{b}/_{a})}}\cdot e^{-j\beta z}\cdot\vec{e}_r        \\
    \vec{S}_{zeit.Mittel} & = \frac{1}{2}\cdot\left[\frac{\hat{I}}{2\pi r}\right]^2\cdot Z_{L}\cdot\vec{e}_z
\end{align*}
\input{Figures/Leitungen_Koaxialleitung.tex}
{\renewcommand*{\arraystretch}{0.2}
    \begin{tabularx}{0.5\columnwidth}{|X|}
        \hline
        \[R=\frac{1}{2\pi\delta\sigma_c}\left[\frac{1}{a}+\frac{1}{b}\right]\] \\
        \hline
        \[L=\frac{\mu}{2\pi}\ln\frac{b}{a}\]                                   \\
        \hline
        \[G=\frac{2\pi\sigma}{\ln(^b/_a)}\]                                    \\
        \hline
        \[C=\frac{2\pi\varepsilon}{\ln(^b/_a)}\]                               \\
        \hline
    \end{tabularx}}



\vspace{1ex}
Für beliebige Leitergeometrie gelten folgende Zusammenhänge:
\[
    LC = \mu\varepsilon \quad \text{und} \quad \frac{G}{C} = \frac{\sigma}{\varepsilon}
\]
Innere Induktivität:
\[
    L_i = \frac{R}{w}
\]

\includegraphics[width=1\columnwidth]{Figures/ZahlentabelleLeitungsparameter.png}