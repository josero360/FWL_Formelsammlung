\section{Smith-Diagramm}

\subsection{Allgemein} \label{sec:Smith_All}


$m$             : Anpassungsfaktor

$s$             : inverser Anpassungsfaktor

$\underline{r}$ : Reflexionsfaktor

$1$             : Anpassungspunkt

\begin{center}
    \begin{align*}
        \Aboxed{r(z)  = r_A \cdot e^{-j2\beta z}}                                              \\
        \Aboxed{Z(z)  = Z_L\cdot\frac{Z_A+jZ_L\cdot\tan(\beta z)}{Z_L+jZ_A\cdot\tan(\beta z)}} \\
        \text{mit} \beta = \frac{2\pi}{\lambda}                                                \\
        \text{auch ohne Quelle gültig!}
    \end{align*}
    \input{Figures/Smithdiagramm_Smithchart.tex}
\end{center}
\begin{align*}
    \underline{z}_n & = \frac{\underline{Z}_n}{Z_L} = \frac{\underline{r}_n + 1}{\underline{r}_n - 1}                                                                                                                \\
    \underline{r}_n & = \frac{\underline{Z}_n-Z_L}{\underline{Z}_n+Z_L}= \frac{\underline{z}_n-1}{\underline{z}_n+1}    = \frac{1-\underline{y}_n}{1+\underline{y}_n} \\
    m               & = \frac{1-|\underline{r}|}{1+|\underline{r}|}                                                                                                   \\
    s               & = \frac{1}{m}
\end{align*}

\subsection{Impedanz/Admetanz umrechnen}
Im Smithchart spiegeln (Phase $\pm 180^{\circ}$/$\pm \pi$)

\subsection{Maxima/Minima bei stehender Welle}\label{sec:max_min_stehende_welle}
Bei \textbf{verlustloser} Leitung:
\begin{align*}
	 & U_{\texttt{max}} = |U_h| \cdot (1+|r(l)|)                            & U_{\texttt{min}} = |U_h| \cdot (1-|r(l)|)                           & \\
	 & I_{\texttt{max}} = \left | \frac{U_h}{Z_L} \right | \cdot (1+|r(l)|) & I_{\texttt{min}} = \left| \frac{U_h}{Z_L} \right | \cdot (1-|r(l)|) &
\end{align*}

Für \textbf{Spannungen}: Abstand von der Last $ z $: \quad $ n=0,1,2,3... $
\begin{align*}
	                                                                                    & z_{\texttt{min}} = \frac{\lambda}{4\pi}(\theta_{rad}+(2n+1)\pi)                             &
	z_{\texttt{max}} = \frac{\lambda}{4\pi} \cdot (\theta_{rad}+2n\pi)                  &                                                                                               \\
	\Aboxed{                                                                            & \text{\textbf{Minima} alle}\: \frac{\lambda}{2} \rightarrow \frac{l}{\lambda}=0.5} \Aboxed{ &
	\text{\textbf{Maxima} alle}\: \frac{\lambda}{4} \rightarrow \frac{l}{\lambda}=0.25} &
\end{align*}
$ \rightarrow $ Schnittpunkte mit der reellen Achse!\\
Strommaxima sind an Spannungsminima und umgekehrt.

\subsection[Von Last zu Quelle]{Lastseite $\rightarrow$ Quelle}
\begin{enumerate}
    \item $Z_L$ ins Diagramm einzeichen
    \item Lastimpedanz bestimmen,
          wenn zB Parallelschaltung etc
    \item Normieren
          \[\underline{z}_a = \frac{\underline{Z}_A}{Z_L} \]
    \item Ins Chart eintragen
    \item Linie vom Mittelpunkt durch $\underline{z}_a$ nach außen

          Ablesen und Notieren:

          $\rightarrow$Relative Länge $\left[\frac{l}{\lambda}\right]$

          $\rightarrow$Relativer Winkel
    \item Kreis einzeichen

          Ablesen und Notiere:

          $\rightarrow$Maxima: rechter Schnittpunkt mit Re-Achse

          $\rightarrow$Minima: linker  Schnittpunkt mit Re-Achse

          $\rightarrow$Rexlexionsfaktor abmessen und aus Skala oben auslesen
    \item Um Leitungslänge im UZS laufen
          $\rightarrow$ Linie vom Mittelpunkt durch neuen Punkt nach außen

          Ablesen und Notieren:

          $\rightarrow$Relativer Winkel
    \item Wenn $\alpha\neq 0$

          $\rightarrow$ Dämpung ausrechen
          $\rightarrow$ Um Faktor nach innen Spiralieren

    \item Dieser Punkt ist $\underline{z}_e$
    \item Eingangsimpedanz ablesen
          \[\underline{Z}_E = \underline{z}_e \cdot Z_L\]
\end{enumerate}

\subsection{Vorgehen mit geg. Eingangswiderstand}
Wenn mit dem Smith-Diagramm gearbeitet wird, liefert dies die Schritte
\ref{Ref L_anfang} und \ref{Bestimmen Z_E}
\begin{enumerate}
	\item Lastimpedanz
	      \[ \underline{Z}_A = \dfrac{1}{\frac{1}{R_A} + j \omega C_A} \]
	\item Reflexion am Leitungsende
	      \[ \underline{r}_A = \underline{r}(z=0) = \dfrac{Z_A - \underline{Z}_L}{Z_A + \underline{Z}_L} \]
	\item Reflexion am Leitungsanfang \label{Ref L_anfang}
	      \[ \underline{r}_E = \underline{r}(z=d) =  \underline{r}_A \cdot e^{-j 2 \beta d}\]
	\item Bestimmung der Impedanz \label{Bestimmen Z_E}
	      \[ \underline{Z}_E = \underline{Z}_L \cdot \dfrac{1 + \underline{r}_E}{1 - \underline{r}_E}\]
	\item Eingangswiderstand
	      \[ \underline{Z}_E = \dfrac{1}{\frac{1}{\underline{Z}_E} + j \omega C_E}\]
\end{enumerate}
\input{Figures/Smithdiagramm_Zusammenschaltungen.tex}  